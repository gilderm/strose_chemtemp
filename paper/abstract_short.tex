\begin{abstract}
To expose students to the cross-disciplinary nature of Computer Science in the STEM fields, an interactive hands-on activity was developed targeting both Computer Science and Chemistry students.  The objective of this project is to give students real-world experience in solving problems in a multi-disciplinary environment.

Computer Science majors are tasked with developing a full end-to-end data acquisition system to be used by chemistry students in their general laboratory course.The project is designed to cover several topics in the Computer Science curriculum including: programming an embedded system to aquire data, integrating different software packages, creating a database, presenting the data, performing data validation, and debugging. The cohort of computer science majors will develop a system to collect and manage thermal data generated during laboratory experiments using an Arduino UNO R4 WiFi, Python programming language, and the PostgreSQL RDBMS. 

Students will work together to identify the requirements of the system to collect and manage thermal data generated during chemical reaction experiments. The objective of this experiment is to determine the optimal (stoichiometric) ratio of two reactants in a chemical reaction using the method of continuous variations. 

In developing this inter-disciplinary project as part of the curriculum, undergraduate students will gain an appreciation of how Computer Science concepts can be leveraged to solve real-world problems.

\end{abstract}
