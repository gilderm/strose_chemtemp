The study of computer science is becoming necessary and fundamental to every academic discipline~\cite{columbia}.  
An understanding of computer science concepts and techniques is critical in order to analyze and 
extract meaningful information from the vast amounts of data generated in fields like biology, healthcare, 
finance, physical sciences, and mathematics.  Due to technological advancements, the ability to store and manage 
large amounts of data is easily accessible across all major STEM fields.  

Comp Sci lead-in to Arduinos as a comp sci teaching tool.  

Arduinos are also becoming increasingly common in undergraduate chemistry laboratories.  They have recently been 
utilized to develop low-cost laboratory equipment including (but not limited to) fluorometers~\cite{bullis}, 
centrifuges~\cite{sadegh}, 
pH sensors~\cite{qutieshat}, 
tensile testers~\cite{arrizabalaga}, and calorimeters~\cite{gomes}.  
In the current study, arduino systems utilizing 
thermistors are developed for use in laboratory experiments that require measurements of temperature.  

Temperature measurements provide important insight into various chemical and physical processes.  
By monitoring the variation of temperature of a sample as it is heated or cooled, characteristic 
transition temperatures such as the freezing/melting point or boiling point can be determined~\cite{atkins}.  
When considering mixtures of varying compositions, freezing point decreases proportionally to concentration.  
Analysis of freezing points over a full range of compositions from pure solvent to pure solute allows for the generation of 
two-component, solid liquid phase diagrams~\cite{martinez,blanchette}.  
Thermochemical analysis through calorimetry can also be utilized to 
determine heats of chemical reaction.  When coupled with the method of continuous variations~\cite{jobs}, stoichiometric 
ratios of chemical reactants or concentration of an unknown reactant can be determined~\cite{vernier,vonderbrink,tatsuoka,mahoney}.  
In order to capture key features in the temperature-time profiles, large numbers of data points are often collected.  
Due to the potentially large data sets generated over reasonably short time periods, prevalence in various introductory through 
advanced chemistry lab courses, and availability of inexpensive thermal probes, thermal analysis activities are a natural 
fit for this interdisciplinary project.

%The introductory chemistry laboratory experiment utilized in this activity allows students to 
%determine optimal ratios of compounds used in chemical reactions.  Students systematically vary the 
%proportions of reactants used in a chemical reaction and identify the proportion that results in 
%maximum evolution of heat (as observed through temperature changes).  



%Temperature measurements provide important insight into various chemical and physical processes.  
%By monitoring the variation of temperature of a sample as it is heated or cooled, characteristic transition 
%temperatures such as the freezing/melting point or boiling point can be determined (REF).  When considering mixtures of varying 
%compositions, freezing point decreases proportionally to concentration.  Analysis of freezing points over a full range of 
%compositions from pure solvent to pure solute allows for the generation of two-component, solid liquid phase diagrams (REF).  
%Thermochemical analysis through calorimetry can also be utilized to determine heats of chemical reaction.  When coupled with 
%the method of continuous variations (REF), stoichiometric ratios of chemical reactants or concentration of an unknown reactant 
%can be determined (REF).  In order to capture key features in the temperature-time profiles, large numbers of data points are often collected.  
%Due to the potentially large data sets generated over reasonably short time periods, prevalence in various introductory through advanced chemistry 
%lab courses, and availability of inexpensive thermal probes, thermal analysis activities are a natural fit for this interdisciplinary project.  (Support with references)

%Two chemistry laboratory activities have been identified for this activity: one in introductory courses and one in upper-level chemistry courses.  
%The first activity allows introductory students to determine optimal ratios of compounds used in chemical reactions by identifying proportions 
%that result in maximum evolution of heat as observed through temperature changes.  
%The second activity is the generation of cooling curves for mixtures of naphthalene and diphenylamine for the purpose of generating a 
%solid-liquid phase diagram.   The frequent monitoring of temperature with time using the Arduino allows chemistry students to 
%capture and identify key physical transitions (freezing points and eutectic points).  

